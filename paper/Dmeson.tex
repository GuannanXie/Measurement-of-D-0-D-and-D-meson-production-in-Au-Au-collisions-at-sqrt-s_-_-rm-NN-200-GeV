% ****** Start of file apssamp.tex ******
%
%   This file is part of the APS files in the REVTeX 4.1 distribution.
%   Version 4.1r of REVTeX, August 2010
%
%   Copyright (c) 2009, 2010 The American Physical Society.
%
%   See the REVTeX 4 README file for restrictions and more information.
%
% TeX'ing this file requires that you have AMS-LaTeX 2.0 installed
% as well as the rest of the prerequisites for REVTeX 4.1
%
% See the REVTeX 4 README file
% It also requires running BibTeX. The commands are as follows:
%
%  1)  latex apssamp.tex
%  2)  bibtex apssamp
%  3)  latex apssamp.tex
%  4)  latex apssamp.tex
%
\documentclass[%
 reprint,
%superscriptaddress,
%groupedaddress,
%unsortedaddress,
%runinaddress,
%frontmatterverbose, 
%preprint,
%showpacs,preprintnumbers,
%nofootinbib,
%nobibnotes,
%bibnotes,
 amsmath,amssymb,
 aps,
%pra,
%prb,
%rmp,
%prstab,
%prstper,
%floatfix,
]{revtex4-1}

\usepackage{graphicx}% Include figure files
\usepackage{dcolumn}% Align table columns on decimal point
\usepackage{bm}% bold math
%\usepackage{hyperref}% add hypertext capabilities
%\usepackage[mathlines]{lineno}% Enable numbering of text and display math
%\linenumbers\relax % Commence numbering lines

%\usepackage[showframe,%Uncomment any one of the following lines to test 
%%scale=0.7, marginratio={1:1, 2:3}, ignoreall,% default settings
%%text={7in,10in},centering,
%%margin=1.5in,
%%total={6.5in,8.75in}, top=1.2in, left=0.9in, includefoot,
%%height=10in,a5paper,hmargin={3cm,0.8in},
%]{geometry}

\begin{document}

\preprint{APS/123-QED}

\title{Measurements of $D^0$, $D^+$ and $D^{*+}$ Meson Production at Mid-rapidity in Au+Au Collisions at $\sqrt{s_{_{\rm NN}}}$ = 200\,GeV by the STAR Experiment}% Force line breaks with \\

\author{Ann Author}
 \altaffiliation[Also at ]{Physics Department, XYZ University.}%Lines break automatically or can be forced with \\
\author{Second Author}%
 \email{Second.Author@institution.edu}
\affiliation{%
 Authors' institution and/or address\\
 This line break forced with \textbackslash\textbackslash
}%

\collaboration{STAR Collaboration}%\noaffiliation

\date{\today}% It is always \today, today,
             %  but any date may be explicitly specified

\begin{abstract}
We report new STAR measurements of $D^+$ and $D^{*+}$ meson production and an improved measurement of $D^{0}$ meson production within mid-rapidity ($|y|<1$) in Au+Au collisions at $\sqrt{s_{_{\rm NN}}}$ = 200\,GeV. The measurements utilize the STAR Heavy Flavor Tracker for topological reconstruction of these mesons through decay channels $D^+\rightarrow K^-\pi^+\pi^+$, $D^{*+}\rightarrow D^{0}\pi^+\rightarrow K^-\pi^+\pi^+$, $D^0\rightarrow K^-\pi^+$ and their charge conjugates. The $D^+/D^0$ and $D^{*+}/D^{0}$ ratios are consistent with PYTHIA8 model calculations in all measured $p_T$ regions and centrality classes. The combined $D$-meson nuclear modification factors $R_{\rm CP}$ and $R_{\rm AA}$ are reported for various centrality bins in Au+Au collisions. We also report the $D^0$ meson yield rapidity distribution within $|y|<1$. Physics implications on charm quarks dynamics in the hot QCD medium will be discussed.

\end{abstract}

\pacs{Valid PACS appear here}% PACS, the Physics and Astronomy
                             % Classification Scheme.
%\keywords{Suggested keywords}%Use showkeys class option if keyword
                              %display desired
\maketitle

%\tableofcontents

\section{Introduction}
\label{sec:intro}

\section{Dataset and Experimental Setup}
\label{sec:dataset}
The measurements reported in this paper utilize the datasets in Au+Au collisions at $\sqrt{s_{_{\rm NN}}}$ = 200\,GeV collected by the STAR experiment during RHIC runs 2014 and 2016. A total statistics of x.x $\times10^9$ minimum bias triggered events were used in these measurements. The trigger condition and STAR subsystems used in the 2014 analysis are the same as those in our previous reported $D^0$ paper~\cite{}. The 


\section{Analysis Detail}

\subsection{$D^0\rightarrow K^-\pi^+$}
\label{sec:ana:D0}

\subsubsection{Signal Reconstruction}
- Run16 $D^0$ signal

\subsubsection{Efficiency Correction}
- Efficiency vs. pT

\subsubsection{Systematic Uncertainties}

\subsection{$D^+\rightarrow K^-\pi^+\pi^+$}
\label{sec:ana:Dplus}
- $D^+$ signal (three pT bins, central/peripheral)
- Efficiency vs. pT

\subsection{$D^{*+}\rightarrow D^0\pi^+\rightarrow K^-\pi^+\pi^+$}
\label{sec:ana:Dstar}
- $D^{*+}$ signal (three pT bins, central/peripheral)
- Efficiency vs. pT

\section{Results}
\label{sec:result}

$D^0$ $p_T$ spectra

$D^+$ $p_T$ spectra

$D^{*+}$ $p_T$ spectra

$D^+/D^0$ ratio compared to PYTHIA

$D^{*+}/D^0$ ratio compared to PYTHIA

$D^0$ $dN/dy$ vs. y compared to PYTHIA

Combined $D$-meson $R_{\rm CP}$

Combined $D$-meson $R_{\rm AA}$

\section{Summary}
\label{sec:summary}

\bibliography{Dmeson}

\end{document}
%
% ****** End of file apssamp.tex ******