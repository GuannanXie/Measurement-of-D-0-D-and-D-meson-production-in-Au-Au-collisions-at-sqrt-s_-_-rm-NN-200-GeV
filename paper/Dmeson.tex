% ****** Start of file apssamp.tex ******
%
%   This file is part of the APS files in the REVTeX 4.1 distribution.
%   Version 4.1r of REVTeX, August 2010
%
%   Copyright (c) 2009, 2010 The American Physical Society.
%
%   See the REVTeX 4 README file for restrictions and more information.
%
% TeX'ing this file requires that you have AMS-LaTeX 2.0 installed
% as well as the rest of the prerequisites for REVTeX 4.1
%
% See the REVTeX 4 README file
% It also requires running BibTeX. The commands are as follows:
%
%  1)  latex apssamp.tex
%  2)  bibtex apssamp
%  3)  latex apssamp.tex
%  4)  latex apssamp.tex
%
\documentclass[%
 reprint,
%superscriptaddress,
%groupedaddress,
%unsortedaddress,
%runinaddress,
%frontmatterverbose, 
%preprint,
%showpacs,preprintnumbers,
%nofootinbib,
%nobibnotes,
%bibnotes,
 amsmath,amssymb,
 aps,
%pra,
%prb,
%rmp,
%prstab,
%prstper,
%floatfix,
]{revtex4-1}

\usepackage{graphicx}% Include figure files
\usepackage{dcolumn}% Align table columns on decimal point
\usepackage{bm}% bold math
%\usepackage{hyperref}% add hypertext capabilities
%\usepackage[mathlines]{lineno}% Enable numbering of text and display math
%\linenumbers\relax % Commence numbering lines

%\usepackage[showframe,%Uncomment any one of the following lines to test 
%%scale=0.7, marginratio={1:1, 2:3}, ignoreall,% default settings
%%text={7in,10in},centering,
%%margin=1.5in,
%%total={6.5in,8.75in}, top=1.2in, left=0.9in, includefoot,
%%height=10in,a5paper,hmargin={3cm,0.8in},
%]{geometry}

\begin{document}

\preprint{APS/123-QED}

\title{Measurements of $D^0$, $D^+$ and $D^{*+}$ Meson Production at Mid-rapidity in Au+Au Collisions at $\sqrt{s_{_{\rm NN}}}$ = 200\,GeV by the STAR Experiment}% Force line breaks with \\

\author{Ann Author}
 \altaffiliation[Also at ]{Physics Department, XYZ University.}%Lines break automatically or can be forced with \\
\author{Second Author}%
 \email{Second.Author@institution.edu}
\affiliation{%
 Authors' institution and/or address\\
 This line break forced with \textbackslash\textbackslash
}%

\collaboration{STAR Collaboration}%\noaffiliation

\date{\today}% It is always \today, today,
             %  but any date may be explicitly specified

\begin{abstract}
We report new STAR measurements of $D^+$ and $D^{*+}$ meson production and an improved measurement of $D^{0}$ meson production within mid-rapidity ($|y|<1$) in Au+Au collisions at $\sqrt{s_{_{\rm NN}}}$ = 200\,GeV. The measurements utilize the STAR Heavy Flavor Tracker for topological reconstruction of these mesons through decay channels $D^+\rightarrow K^-\pi^+\pi^+$, $D^{*+}\rightarrow D^{0}\pi^+\rightarrow K^-\pi^+\pi^+$, $D^0\rightarrow K^-\pi^+$ and their charge conjugates. The $D^+/D^0$ and $D^{*+}/D^{0}$ ratios are consistent with PYTHIA8 model calculations in all measured $p_T$ regions and centrality classes. The combined $D$-meson nuclear modification factors $R_{\rm CP}$ and $R_{\rm AA}$ are reported for various centrality bins in Au+Au collisions. We also report the $D^0$ meson yield rapidity distribution within $|y|<1$. Physics implications on charm quarks dynamics in the hot QCD medium will be discussed.

\end{abstract}

\pacs{Valid PACS appear here}% PACS, the Physics and Astronomy
                             % Classification Scheme.
%\keywords{Suggested keywords}%Use showkeys class option if keyword
                              %display desired
\maketitle

%\tableofcontents

\section{Introduction}
\label{sec:intro}

The heavy-ion program at the Relativistic Heavy Ion Collider (RHIC) and Large Hadron Collider (LHC) allows to study the strong interactions and Quantum Chromodynamics (QCD) at high temperature and density. Under such extreme conditions, lattice QCD calculations predict a transition to a strongly-coupled Quark-Gluon Plasma (sQGP) occurs. Over the past decades RHIC and LHC observed strong collective flow and the large suppression at high transverse momentum ($p_{T}$) in central collisions for various observed hadrons including light hadrons and multi-strange-quark hadrons, which is a strong evidence for the QGP been formed.

Heavy quarks ($c$,$b$) are created predominantly through the initial hard scatterings in the early stage of the collision. In contrast, the thermal production is negligible. Due to the large mass, their production time is shorter than the formation time of the QGP at RHIC. Therefore, the heavy quarks experience the whole evolution of the hot and dense medium and are an important tool for studying the properties of the QGP. 

During the propagation in the medium, heavy quarks lose energy via inelastic (gluon radiation) or elastic scatterings (collisional processes). With a smaller color coupling factor with respect to gluons, quarks are expected to lose less energy than gluons. In addition, the ``dead cone effect" is expected to reduce the small-angle gluon radiation for heavy quarks compare to both gluons and light quarks. Above effects attenuate the medium effect for heavy quarks while the other mechanisms such as the in-medium hadron formation and dissociation would make another stronger effect, characterized by smaller formation time than light-flavor hadrons, especially on the low to intermedius $p_T$ range. Low-$p_T$ heavy quarks can also achieve some extent thermalization and participate in the collective expansion of the system as a consequence of multiple interactions with the medium. The modification to their production in transverse momentum due to energy loss, radial flow and hadronization mechanisms is sensitive to heavy quark dynamics in the partonic sQGP phase. 

The energy loss and the heavy-quark hadronization dynamics can be characterized by the nuclear modification factor $R_{\rm AA}$ and $R_{\rm CP}$, defined as
\begin{equation}
  R_{\rm AA}(p_T) = \frac{1}{\langle T_{\rm AA}\rangle} \frac{dN_{\rm AA}/dp_{T}}{d\sigma_{pp}/dp_{T}},
\label{equ:equation1}
\end{equation}
\begin{equation}
  R_{\rm CP} = \frac{d^2N/dp_{T}dy}{N_{\rm bin}} |_{\rm cen\ } \times \frac{N_{\rm bin}} {d^2N/dp_{T}dy} |_{\rm peri }.
\label{equ:equation2}
\end{equation}
where $dN_{\rm AA}/dp_T$ and $d\sigma_{pp}/dp_T$ are particle production yield and cross section in A+A and $p$+$p$ collisions, respectively. $T_{\rm AA}=\langle N_{\rm bin}\rangle/\sigma_{pp}^{\rm inel}$ represent the nuclear thickness function and often calculated using a Monte-Carlo Glauber model, where $\langle N_{\rm bin}\rangle$ is the average number of binary collisions and $\sigma_{pp}^{\rm inel}$ is the total inelastic $p$+$p$ cross section. $R_{\rm AA}$ present the ratio between $N_{\rm bin}$-normalized yields in A+A collisions and $p$+$p$ collisions, and $R_{\rm CP}$ is the normalized yield ratio between central and peripheral collisions.

Recent measurements from the Solenoidal Tracker At RHIC (STAR) show that in the central (0--10\%) collisions $D^0$ meson suffer a significant suppression at the high $p_T$ range ($p_T$ $>$ 4 GeV/$c$). The suppression level is similar with the light hadrons, which reaffirm the strong interactions between charm quarks and the medium. At low $p_T$, the nuclear modification factors show a characteristic structure qualitatively consistent with the model expectations that charm quarks gain sizable collective motion during the medium evolution. More differential measurements on $D^0$ meson and other charm hadron species with better precisions can offer new constraints on the model calculations and the total charm cross sections.

In this article, we report the measurements of the centrality dependence of $D^{+}$ and $D^{*+}$ meson and an improved $D^0$ meson production at mid-rapidity ($|y|$\,$<$\,1) in Au+Au collisions at $\sqrt{s_{_{\rm NN}}} = 200$\,GeV. The measurements are conducted at STAR experiment utilizing the Heavy Flavor Tracker (HFT) detector for topological reconstruction. The mesons are reconstructed through the decay channels $D^+\rightarrow K^-\pi^+\pi^+$, $D^{*+}\rightarrow D^{0}\pi^+\rightarrow K^-\pi^+\pi^+$, $D^0\rightarrow K^-\pi^+$ and their charge conjugates. The paper is organized in the following order: In Sec.~\ref{sec:dataset}, we describe the detector setup and dataset. In Sec.~\ref{sec:ana}, we present the analysis details including the topological reconstruction together with the efficiency corrections and systematic uncertainties for $D^0$ meson in sec.~\ref{sec:ana:D0}, $D^{+}$ meson in sec.~\ref{sec:ana:Dplus} and $D^{*+}$ meson in sec.~\ref{sec:ana:Dstar} respectively. We present our measurement results and physics discussions in Sec.~\ref{sec:result}. Finally, we end the paper with a summary in Sec.~\ref{sec:summary} .

\section{Dataset and Experimental Setup}
\label{sec:dataset}
The measurements reported in this paper utilize the datasets of Au+Au collisions at $\sqrt{s_{_{\rm NN}}}$ = 200\,GeV collected by the STAR experiment during RHIC runs 2014 and 2016. A total statistics of 1.9 $\times10^9$ minimum bias triggered events were used in these measurements. The trigger condition and STAR subsystems used in the 2014 analysis are the same as those in our previous reported $D^0$ paper~\cite{}. Charged tracks are reconstructed via a combination of the Time Projection Chamber (TPC) and the Heavy Flavor Tracker (HFT) which yields a track-pointing-resolution to the collision vertex of $<55\mu m$ for charged kaon tracks at a momentum ($p$) of 750\,MeV. 


\section{Analysis Detail}
\label{sec:ana}

\subsection{$D^0\rightarrow K^-\pi^+$}
\label{sec:ana:D0}

\subsubsection{Signal Reconstruction}
- Run16 $D^0$ signal

\subsubsection{Efficiency Correction}
- Efficiency vs. pT

\subsubsection{Systematic Uncertainties}

\subsection{$D^+\rightarrow K^-\pi^+\pi^+$}
\label{sec:ana:Dplus}
- $D^+$ signal (three pT bins, central/peripheral)
- Efficiency vs. pT

\subsection{$D^{*+}\rightarrow D^0\pi^+\rightarrow K^-\pi^+\pi^+$}
\label{sec:ana:Dstar}
- $D^{*+}$ signal (three pT bins, central/peripheral)
- Efficiency vs. pT

\section{Results}
\label{sec:result}

$D^0$ $p_T$ spectra

$D^+$ $p_T$ spectra

$D^{*+}$ $p_T$ spectra

$D^+/D^0$ ratio compared to PYTHIA

$D^{*+}/D^0$ ratio compared to PYTHIA

$D^0$ $dN/dy$ vs. y compared to PYTHIA

Combined $D$-meson $R_{\rm CP}$

Combined $D$-meson $R_{\rm AA}$

\section{Summary}
\label{sec:summary}

\bibliography{Dmeson}

\end{document}
%
% ****** End of file apssamp.tex ******
